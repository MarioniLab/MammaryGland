\documentclass[oneside]{amsart}
\usepackage{geometry}
\usepackage{times}


%% --for bibtex
\bibliographystyle{unsrt}
%\usepackage{cite}
\usepackage{natbib}

\usepackage{amsmath}
\usepackage{amssymb}
\usepackage{amsfonts} %\mathbb
\usepackage{bm}
\usepackage{lineno}

\usepackage{subcaption}
\usepackage{graphicx}
\usepackage{hyperref}

\usepackage{siunitx}

\hypersetup{colorlinks=true,linkcolor=black,citecolor=black,urlcolor=black}
\urlstyle{same}

\newcommand{\fixme}[1]{\textit{\textcolor{red}{Fixme: #1}}}
\newcommand{\comment}[1]{\textit{\textcolor{blue}{Comment: #1}}}
%\DeclareSIUnit\Molar{\textsc{m}} % add molar as captial m to siunitx

\title{Single-cell RNA-sequencing reveals cellular dynamics of the developing mammary gland}
\author{Karsten Bach, Sara Pensa, Walid T. Khaled, John C. Marioni}
\begin{document}
\begin{abstract}
    Bla bla bla
\end{abstract}

\maketitle
\tableofcontents
\newpage
\section{Introduction}

\section{Results}
%------------------------------ Section 1 -------------------------------------------------------
\subsection{Unbiased identification of mammary epithelial cells\label{sec:cluster}}
\begin{itemize}
    \item Experimental Setup, 4 stages of the development of an adult gland
    \item Clustering into major cell types in the mammary gland
    \item Luminal cells: Hormone sensing (HS) (progenitor and differentiated), Non-hormone sensing (NHS) progenitor and secretory cells
    \item Basal cells: MaSCs, PROCR\textsuperscript{+} MaSCs and differentiated myoepithelial cells
\end{itemize}

%------------------------------ Figure 1 -------------------------------------------------------
\begin{figure}[h]
\begin{subfigure}{0.45\linewidth}
\centering %\textbf{a}
\fbox{\parbox{0.9\linewidth}{(Experimental overview)}}
%\includegraphics[height=5cm]{../figures/ROC_Staurosporine.pdf}
\caption{}
\end{subfigure}
\begin{subfigure}{0.45\linewidth}
%\centering %\textbf{a}
    \fbox{\parbox{0.9\linewidth}{(tSNE plot colored by Condition)}}
%\includegraphics[height=5cm]{../figures/ROC_Panobinostat.pdf}
\caption{}
\end{subfigure}
\begin{subfigure}{0.45\linewidth}
\centering %\textbf{a}
\fbox{\parbox{1\linewidth}{(Plot to illustrate clusters and marker genes)}}
%\includegraphics[height=5cm]{../figures/ROC_ATP.pdf}
\caption{}
\end{subfigure}
    \caption{Unbiased identification of mammary epithelial cells}
      \label{fig:f1}
  \end{figure}

\newpage
%------------------------------ Section 2 -------------------------------------------------------
\subsection{The luminal compartment is defined by a continuous spectrum of differentiation\label{sec:LumDiff}}
\comment{Here I would start with only looking at virgin and pregnancy arguing that here we would expect all celltypes. 
I would then describe the parity effect in the next section. Maybe there is a more elegant way of disentagling the two.}
\begin{itemize}
    \item Basal cells are separate from the luminal compartment, suggesting either:
	\begin{itemize}
	    \item This is a rare event 
	    \item Intermediate stages are highly unstable and the transdifferentation is rapid
	\end{itemize}
    \item In the virgin gland there is one ``branch'' of differentiation with NHS progenitors on one side and differentiated HS cells on the other end
    \item In pregnancy another branch derives from the NHS progenitors with differentiated secretory cells at the tip
    \item We conclude that there is a luminal hierarchy with NHS progenitors that can give rise to both HS progenitors/differentiated cells and secretory cells
\end{itemize}

%------------------------------ Figure 2 -------------------------------------------------------
\begin{figure}[h]
\begin{subfigure}{0.45\linewidth}
\centering %\textbf{a}
\fbox{\parbox{0.9\linewidth}{(3D scatterplot of DiffusionMap V/P)}}
%\includegraphics[height=5cm]{../figures/ROC_Staurosporine.pdf}
\caption{}
\end{subfigure}
\begin{subfigure}{0.45\linewidth}
%\centering %\textbf{a}
    \fbox{\parbox{0.9\linewidth}{(Only luminal cells in virgin and pregnancy)}}
%\includegraphics[height=5cm]{../figures/ROC_Panobinostat.pdf}
\caption{}
\end{subfigure}
\begin{subfigure}{0.45\linewidth}
\centering %\textbf{a}
\fbox{\parbox{1\linewidth}{(Multiple plots to show gene expression pattern along the two branches)}}
%\includegraphics[height=5cm]{../figures/ROC_ATP.pdf}
\caption{}
\end{subfigure}
\caption{Differentiation hierarchy in the mammary gland}
      \label{fig:f2}
  \end{figure}
  

\newpage
%------------------------------ Section 3 -------------------------------------------------------
\subsection{Parity exerts cell type specific effects on the transcriptional landscape of luminal cells\label{sec:ParityEffect}}
\begin{itemize}
    \item Effect is stronger on luminal than on basal cells (\% variance explained?) [as has been shown before on epigenitic level]
    \item \fixme{Is it worth mentioning that the lactation sample FACS profile has completely lost its ``basal/luminal'' appearence, yet contains all cell types}
    \item DE genes between e.g. HS differentiated in involution versus virgin
    \item A ``new'' cell type arises after pregnancy that most likely derives from NHS progenitors 
    \item Highlight importance of this cell type as it sits at the top of the hierarchy
\end{itemize}


%------------------------------ Figure 4 -------------------------------------------------------
\begin{figure}[h]
\begin{subfigure}{0.45\linewidth}
\centering %\textbf{a}
\fbox{\parbox{0.9\linewidth}{(Condition factor explains more variance in the luminal compartment)}}
%\includegraphics[height=5cm]{../figures/ROC_Staurosporine.pdf}
\caption{}
\end{subfigure}
\begin{subfigure}{0.45\linewidth}
%\centering %\textbf{a}
    \fbox{\parbox{0.9\linewidth}{(Cluster 4 is closest to 5 and is at the top of the hierarchy after parity)}}
%\includegraphics[height=5cm]{../figures/ROC_Panobinostat.pdf}
\caption{}
\end{subfigure}
\begin{subfigure}{0.45\linewidth}
\centering %\textbf{a}
\fbox{\parbox{1\linewidth}{(Gene expression differences between 4 and 5)}}
%\includegraphics[height=5cm]{../figures/ROC_ATP.pdf}
\caption{}
\end{subfigure}
\caption{Parity exerts cell type specific effects on the transcriptional landscape of luminal cells}
      \label{fig:f3}
  \end{figure}

\section{Discussion}
\section{Methods}

\clearpage

%% --for biblatex
\bibliographystyle{unsrt}
\bibliography{nonparatpp}

%% --for biblatex
%\printbibliography

\end{document}

