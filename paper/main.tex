\documentclass[titlepage, 12pt, oneside]{amsart}
\usepackage[backend=bibtex,
	    natbib=true,
	    autocite=superscript,
	    style=nature,
	    sorting=none,
	    isbn=false]{biblatex} 
\addbibresource{references.bib}
\usepackage{geometry}
\usepackage{amssymb,amsmath,amsaddr}
\usepackage{siunitx}
\usepackage{fixltx2e} % provides \textsubscript
\usepackage{hyperref}
\usepackage{color}
\usepackage{setspace}
\PassOptionsToPackage{usenames,dvipsnames}{color} % color is loaded by hyperref
\hypersetup{unicode=true,
            pdfborder={0 0 0},
            breaklinks=true}
\urlstyle{same}  % don't use monospace font for urls
\setlength{\parindent}{0pt}
\setlength{\parskip}{6pt plus 2pt minus 1pt}
\setcounter{secnumdepth}{0}

%---------------- fake figures ------------------------
\newcommand{\fakefigure}[1]% #1 = label name
{\refstepcounter{figure}\label{#1}}
%------------------------------------------------------

%---------------- fake tables ------------------------
\newcommand{\faketable}[1]% #1 = label name
{\refstepcounter{table}\label{#1}}
%------------------------------------------------------

%---------------- editing------------------------
\newcommand{\fixme}[1]{\textit{\textcolor{red}{Fixme: #1}}}
\newcommand{\comment}[1]{\textit{\textcolor{blue}{Comment: #1}}}
%------------------------------------------------------

\begin{document}

\begin{titlepage}
\large
\textbf{Differentiation dynamics of mammary epithelial cells revealed by
single-cell RNA-sequencing}
\normalsize

\small
Karsten Bach\textsuperscript{1,2,5}, Sara Pensa\textsuperscript{1,5},
Marta Grzelak\textsuperscript{2,5}, James Hadfield\textsuperscript{2,5},
David J. Adams\textsuperscript{3,5}, John C.
Marioni\textsuperscript{2,3,4}, Walid T. Khaled\textsuperscript{1,5 }

1. Department of Pharmacology, University of Cambridge, Cambridge, UK\\
2. Cancer Research UK Cambridge Institute, University of Cambridge,
Cambridge, UK\\
3. Wellcome Trust Sanger Institute, Wellcome Genome Campus, Hinxton,
Cambridge, UK\\
4. European Bioinformatics Institute, European Molecular Biology
Laboratory, Hinxton, UK\\
5. Cancer Research UK Cambridge Cancer Centre, Cambridge, UK

Correspondence to: Walid T. Khaled
(\href{mailto:wtk22@cam.ac.uk}{\emph{wtk22@cam.ac.uk}}) or John C.
Marioni (\href{mailto:marioni@ebi.ac.uk}{\emph{marioni@ebi.ac.uk}})

\normalsize
\textbf{Keywords: Single cell RNAseq, Development, Mammary gland }
\end{titlepage}
\onehalfspacing
\textbf{Abstract:}
\textbf{The mammary gland is a unique organ as it undergoes most of its development during puberty and adulthood.
Characterising the hierarchy of the various mammary epithelial cells and how they are regulated in response to gestation, lactation and involution is important for understanding how breast cancer develops.
Recent studies have used numerous markers to enrich, isolate and characterise the different epithelial cell compartments within the adult mammary gland.
However, in all of these studies only a handful of markers were used to define and trace cell populations.
Therefore, there is a need for an unbiased and comprehensive description of mammary epithelial cells within the gland at different developmental stages.
To this end we used single cell RNA sequencing (scRNAseq) to determine the gene expression profile of individual mammary epithelial cells across four adult developmental stages; nulliparous, mid gestation, lactation and post weaning (full natural involution).
Our data from 25,010 individual cells identifies 8 distinct mammary epithelial cell populations and allows their hierarchical structure across development to be charted.
Interestingly, the effect of gestation and lactation appeared to be more pronounced for some cell types.
For example, our analysis revealed a cluster of luminal progenitor cells in post involution glands, which is distinct from progenitors found in nulliparous glands.
The data also showed that few clusters could be fully characterized by a single marker gene.
We argue instead that the epithelial cells -- especially in the luminal compartment -- should rather be conceptualized as being part of a continuous spectrum of differentiation.
This view highlights the plasticity of the tissue and might help to explain some of the conflicting results from lineage tracing studies.}
\newpage

\section{Introduction}
The purpose of the mammary gland is to provide nourishment and passive immunity for the young until they are capable of feeding themselves.
From a developmental biology perspective, the mammary gland is a unique organ as it undergoes most of its development during puberty and adulthood\autocite{Watson2008,Hennighausen2005,Hennighausen1998,Inman2015}.
In the pre-pubertal mouse the mammary gland consists of a rudimentary epithelial ductal structure embedded within a mammary fat pad, which is connected to the nipple\autocite{Mikkola2006,Hens2005}.
At the onset of puberty and in response to hormonal changes, the rudimentary ductal structure will proliferate and migrate to fill the entire mammary fat pad, leaving a developed network of ductal structures that later serve as channels for milk transport during lactation.
At the onset of pregnancy a highly proliferative stage is initiated, characterised by further ductal side-branching and widespread lobuloalveolar development\autocite{Watson2008}.
Differentiation of the epithelial cells within alveoli prepares the gland for milk production and secretion.
Towards the end of pregnancy the gland is extremely dense and primarily occupied by epithelial cells and very little fat.
This morphology is largely maintained throughout lactation.
However, in response to cessation of suckling the gland undergoes involution, which is characterised by extensive cell death and tissue remodelling \autocite{Inman2015,Watson2006}.
Towards the end of involution the gland reaches a morphology resembling that of glands prior to pregnancy and subsequent pregnancies will trigger the same chain of events.

Recent efforts have focused on the identification and characterisation of the various mammary epithelial cell lineages within the gland that contribute to this developmental homeostasis.
Pioneering fat pad transplantation studies nearly 70 years ago were the first to demonstrate the regenerative and differentiation capacity of small numbers of cells\autocite{Faulkin1960,Daniel197,Smalley1998}.
More recently the use of cell surface markers coupled with flow cytometry has been used to enrich for various progenitor and stem cell compartments\autocite{Smalley1998,Stingl2006,Shackleton2006,Asselin2007} and showed the imbalance of such cell populations results in cellular transformation and subsequently breast cancer\autocite{Lim2009,Molyneux2010}.
Other studies, inspired by breast cancer transcriptomic profiling, have identified transcriptional regulators of mammary epithelial cell types such as \textit{GATA3} in luminal cells\autocite{Asselin2007,Kouros2006}.
More recently, elegant lineage-tracing studies used key markers to address the contribution of each lineage to adult mammary epithelial cell homeostasis\autocite{Inman2015}.
However, in all of these studies only a handful of markers and genes were used to define the cellular hierarchy of the mammary epithelial cells, with a principal focus on the nulliparous developmental stage.
Therefore, to properly understand its changing role throughout life, there is a need for an unbiased and comprehensive characterisation of mammary epithelial cell compartments at different developmental stages.

Here we used single cell RNA sequencing (scRNAseq) to map the cellular dynamics of mammary epithelial cells across four adult developmental stages; nulliparous, mid gestation, lactation and post weaning (full natural involution).
Our data from 25,010 individual cells, identifies 15 distinct cell populations within the gland and allows their hierarchical structure across developmental time points to be charted.

\section{Results}

\textbf{Single cell RNA sequencing identifies 15 clusters of mammary epithelial cells}

First, we isolated mammary epithelial cells from four developmental time points; nulliparous (NP), day 14.5 gestation (G), day 6 lactation (L) and 11 days post natural involution (PI) (Supplementary Figure 2).
For each time point we sorted mammary epithelial cells based on the EpCAM cell surface marker from two independent mice (\autoref{F1}a).
All samples were then prepared for single-cell RNA-sequencing using the 10x Chromium platform\autocite{Zheng2017}.
Following quality control (materials and methods), this yielded an average of 6,175 unique transcripts and 2,118 genes detected from 25,010 cells (4,223 in NP, 5,826 in G, 9,319 in L and 5,642 in P) (Supplementary Figure 3).
Visual inspection of the data using t-distributed stochastic neighbor embedding (t-SNE) suggested that although there is grouping of cells by time-point, there are also other factors that underlie structure within the dataset (\autoref{F1}b and Supplementary Figure 4).
First, we dissected the global structure by unsupervised clustering using a shared-nearest-neighbor clustering approach.
This resulted in a coarse clustering into 13 groups (Supplementary Figure 5a).
In a second step we applied hierarchical clustering to each of the identified groups to further resolve the cellular heterogeneity (\autoref{F1}c, Supplementary Figure 5b).
After removal of endothelial and immune cell clusters this resulted in a total of 15 clusters of mammary epithelial cells.
Differential gene expression analysis allowed us to further characterise the clusters and infer putative identities (\autoref{F1}d, \autoref{T1} and Supplementary Table 1).
Notably, some clusters mainly represent the same putative cell type at different developmental time points (indicated by the NP, G, L or PI suffix, in the following the suffix might be omitted to refer to a group of clusters). 
Based on the expression of \textit{Krt18,} \textit{Krt8, Krt5, Krt14} and \textit{Acta2} we noted that 11 clusters show a luminal profile (Lp, Hsp, Hsd, Avp \& Avd clusters) and 4 a basal profile (Bsl, Myo \& Prc clusters) (\autoref{F1}D, Supplementary Fig 2b and 2c).
The Avd and Avp clusters express high levels of beta-casein, \textit{Csn2}, and are exclusively composed of cells from the gestation and lactation time-points, suggesting that these cells are secretory alveolar cells (\autoref{F1}d).
Hsd and Hsp on the other hand represent the clusters with the highest expression of hormone receptors, suggesting hormone sensing function.
Lp expresses high levels of the luminal progenitor marker \textit{Aldh1a3} suggesting progenitor function, while Hsp and Avp also express progenitor markers but to a lesser extent.
Myo shows characteristics of myoepithelial cells such as high levels of \textit{Acta2}, \textit{Oxtr} and \textit{Krt15} (\autoref{F1}d).

\textbf{Reconstruction of the luminal differentiation hierarchy}

We focused on cells from the NP and G time-points to investigate mammary epithelial differentiation states of the gland.
These differentiation states and the transitions between them can be computationally reconstructed using diffusion maps.
Briefly, the method embeds the data in a low-dimensional space, where distances between cells represent the progression through a gradual but stochastic process such as differentiation.
In diffusion maps constructed from all epithelial cells, we observed a clear segregation between the luminal and basal clusters, with virtually no transition states between the two (\autoref{F2}a).
This supports the hypothesis that during normal tissue homoeostasis the two lineages are largely self-maintained, which is in agreement with the majority of the lineage tracing studies\autocite{VanKeymeulen2011,VanAmerongen2012,Davis2016}.
In contrast, the luminal compartment showed a distinct structure, with gradual transitions between different clusters and cells originating from a common origin (\autoref{F2}b).
We confirmed the robustness of this bifurcation by verifying that it was present when different methods of feature selection, algorithms for trajectory inference and down-sampling were employed (Supplementary Figure 6).

The expression of the progenitor marker \textit{Aldh1a3} gradually decreased as cells progressed away from their common origin (\autoref{F2}c), which was largely composed of Lp.
We further noted that the left arm of the differentiation trajectory terminates at Avd and shows increasing expression of \textit{Csn2} and \textit{Glycam1} (\autoref{F2}c), consistent with a secretory phenotype.
Between Lp and Avd we found cells from Avp (\autoref{F2}b).
On the right arm of the differentiation trajectory, cells in Lp transitioned to Hsp and Hsd, during which the expression of \textit{Cited1} and \textit{Pgr} increase suggesting that this branch represents differentiation towards hormone-sensing luminal cells (\autoref{F2}c).

Being confident that the diffusion map recapitulates the luminal differentiation process, we then computationally inferred the two branches and ordered the cells according to their progression through ``pseudotime''\autocite{Haghverdi2016} (see materials and methods for further explanation) (\autoref{F3}a).
This allows us to identify genes whose expression changes during the process of differentiation.
We found 456 genes that showed a pseudotime-dependent expression with the same directionality along both differentiation trajectories (Supplementary Table 2).
These included genes associated with known progenitor characteristics such as \textit{Aldh1a3} and \textit{Tspan8} as well as transcription factors that have not previously been associated with luminal differentiation such as \textit{Creb5}, \textit{Hmga1} and \textit{Fosl1} (\autoref{F3}b).
Interestingly, \textit{Hmga1} is a known chromatin remodeller with reprogramming activity that has been implicated in basal-like breast cancer\autocite{Shah2012,Shah2013}.
In addition, we identified 1005 genes with branch-specific expression patterns (\autoref{F3}d, \autoref{F3}e, Supplementary Table 3).
We further clustered the gene expression trend on each of the two branches and identified sets of genes that change over pseudotime (\fixme{Supplementary Figure 7}).
On the alveolar branch we found two clusters of genes that increase in expression during differentiation, one that is switched on early and another that is activated later in the final stages of differentiation.
The early cluster is enriched for genes involved in fatty-acid oxidation and lipid biosynthetic processes (e.g. \textit{Lipa}, \textit{Dbi} and \textit{Acox1}) .
The second, smaller group of genes that is switched on during the late stages of differentiation was enriched for genes involved in fatty-acid transport and lipid homeostasis (e.g. \textit{Olah}, \textit{Cd36}).
Genes involved in cell division and translation show a transient phase of up-regulation during the alveolar differentiation process (e.g. \textit{Aurka}, \textit{Cenpa}, \textit{Cdk6}), whereas genes regulating cell shape and morphogenesis of a branching epithelium are repressed early on (e.g. \textit{Vdr} and \textit{Areg}).
During the process of differentiation towards hormone-sensing luminal cells, we identified two broad clusters of genes that gradually increase or decrease their expression.
Genes with increasing expression are involved in hormone metabolic processes and the regulation of morphogenesis of an epithelium (e.g. \textit{Vdr}, \textit{Esr1}, \textit{Pgr} and \textit{Msx2}).
Amongst the branch-specific transcription factors we found for example \textit{Runx1, Tox2} and \textit{Bhlhe41} (also known as \textit{Sharp1}) to be transcribed during differentiation towards the hormone-sensing lineage (\autoref{F3}c) and \textit{Elf5, Foxs1} and \textit{Ehf} in the secretory lineage (\autoref{F3}d).
\textit{Runx1} is a known repressor of \textit{Elf5} and its deletion has been shown to be deleterious for ductal morphogenesis\autocite{VanBragt2014}.
The expression of the known progenitor master regulator \textit{Elf5} is maintained and further increased during secretory differentiation suggesting that its transcriptional level is fine-tuned in luminal progenitors (\autoref{F3}d).

\textbf{Parity primes luminal progenitors towards the alveolar fate}

Our data illustrates how the cellular composition of the gland changes during the pregnancy cycle.
The luminal compartment shifts from giving rise to mainly hormone sensing cells to producing alveolar, milk-producing cells during pregnancy and lactation. 
The basal compartment on the other hand differentiates to produce oxytocin-sensing, myoepithelial cells that enable duct contraction and milk secretion during lactation.
At the end of lactation the gland then reverts back to a stage that resembles the pre-pregnancy state.
However, we found that in particular the luminal compartment differs from its nulliparous counterpart (\autoref{F1}b and \autoref{F1}d).
Interestingly, the effect of parity is most pronounced for the less differentiated and more progenitor-like cell types (\autoref{F1}c, Supplementary Figure 8).
To ensure that Lp-PI still represents the progenitor population in the post-parous gland, we identified genes that distinguish Lp-NP from the rest of the luminal compartment to see if they are also characteristic for the proposed post-parity progenitor population Lp-PI.
Indeed, we find genes that are differentially expressed between Lp-NP and the rest of the luminal compartment to show the same trend between Lp-PI and the luminal compartment (\autoref{F4}a).
In a similar manner we can distinguish Lp-PI from the rest of the PI gland in a principal component analysis (PCA) using the identified progenitor genes (Supplementary Fig 8b and c).
From this we conclude, that Lp-PI indeed represents the post-parity luminal progenitor population.
Interestingly, genes that were up-regulated in Lp-PI compared to Lp-NP were significantly enriched for pathways that are involved in the immune response and lactation (\autoref{F4}b, Supplementary Fig 8d).
These included genes that play roles in various processes during lactation, e.g. milk-proteins (\textit{Csn2}, \textit{Lalba}), lipases (\textit{Lipa}), proteins involved in milk secretion (\textit{Xdh}, \textit{Cd36}) and transcriptional regulators of lactation (\textit{Cidea}).
Of note, the genes of the casein locus (\textit{Csn2}, \textit{Csn1s1}, \textit{Csn1s2a}, \textit{Csn3}) have previously been reported to be up-regulated in the parous gland, most likely due to changes in chromatin accessibility\autocite{Dos2015,Rijnkels2013}.
However, it has not been shown before that this effect is confined to the progenitor population of the luminal compartment.
Some of the genes that are involved in the immune response are known regulators of the involution process such as \textit{Ctsc}, \textit{Tgfb3} or \textit{Mfge8}.
The up-regulation of these genesets was also present in the Hsp and in some cases even the differentiated Hsd cluster, yet the effect remained strongest in the progenitor cells (\autoref{F4}c-d).
Finally, we compared the differentiation processes of the luminal compartment of the parous gland to the nulliparous gland by mapping the luminal cells from PI to the trajectory of NP and G cells.
Interestingly, the Hsp and Hsd cluster generally maintain the position of their nulliparous counterpart in the differentiation hierarchy whereas the Lp-PI cluster is stretched out from the origin down towards the alveolar branch, suggesting that these are biased towards the alveolar fate (\autoref{F4}e).
Together, the data suggest that luminal progenitor cells maintain memory of having undergone gestation and involution.
This memory could potentially prime progenitor cells towards the alveolar fate to facilitate alveologenesis in subsequent pregnancies.

\section{Discussion}

We have reported here the use of single cell RNA sequencing to comprehensively map the transcriptomes of thousands of mammary epithelial cells across four developmental time-points.
Our analysis identified 15 clusters of epithelial cells, some of which are only present during specific developmental stage (e.g. Avd and Lp-PI).
This study provides a rich dataset that can be mined online (see link in materials and methods) to identify marker genes and lineage specific promoters that can be used to trace populations of cells \textit{in vivo}.
We note, however, that only some of the clusters can be fully characterized by a single marker gene.
Instead we argue that the epithelial cells - especially in the luminal compartment - should rather be conceptualized as being part of a continuous spectrum of differentiation as visualised in \autoref{F2}.
This view highlights the plasticity of the tissue and might help to explain some of the conflicting results from lineage tracing studies\autocite{Inman2015}.
In this study we could not provide any evidence for contribution of a putative mutlipotent stem cell to the day-to-day homoeostasis of the gland.
We do note, however, that the Prc cluster expresses low levels of both luminal and basal markers, high levels of the stem cell marker \textit{Procr}\autocite{Wang2015} and high levels of the luminal progenitor marker \textit{Notch3}\autocite{Lafkas2013} (\autoref{F1}d).
Yet, our data does not support a central role of this cluster in the day-to-day homeostasis of the gland. 
In addition, cells from the Prc cluster also express some but not all markers of pericytes and we thus cannot exclude that these are non-epithelial cells. 
Based on the gene expression data presented here, the luminal compartment appears to have one common progenitor population (Lp).
Lp gives rise to intermediate states of either hormone-sensing (Hsp) or secretory (Avp) cells that express differentiation markers as well as progenitor markers and appear to represent meta-stable states on the differentiation path from luminal progenitors towards fully differentiated cells (Hsd).
Furthermore, we characterised gene expression patterns along the differentiation hierarchy.
Here we identified genes that are lineage-specific, thus enabling us to disentangle the transcriptional events that regulate differentiation of the luminal compartment.
By analysing the mammary gland at various stages of development we were also able to describe the effects of parity at cellular resolution.
We found that the luminal progenitor compartment undergoes lasting changes at the transcriptional level.
This is especially interesting in the light of the protective effect of pregnancies against breast cancer and the role of luminal progenitors as cell of origin.
The majority of the changes were related to pathways involved in immunity and lactation, suggesting that, in particular, the luminal progenitors maintain a memory of gestation and involution.
It is reasonable to assume that Lp-PI overlaps with the previously described parity-induced mammary epithelial cells (PI-MECs)\autocite{Wagner2002}.

In summary, the study provides a novel view of mammary gland development.
This unbiased approach helps support some previously formed hypotheses in the mammary gland field and describes differentiation processes at a high cellular resolution.
The dataset will be a useful resource for future studies that aim to understand the relationship of the different cell types in the gland and how breast cancer develops and progresses.

\section{Figure legends}

%---------------------------------------------Figure 1--------------------------------------------------------------
\textbf{Figure 1. Single cell RNA sequencing identifies 15 clusters of mammary epithelial cells }
\fakefigure{F1}
\textbf{(a)} Schematic diagram highlighting the experimental setup for isolating and sequencing the RNA of single cells using the 10X chromium system.
\textbf{(b)} t-SNE plot of 25,010 cells visualizes general structure in the data.
Cells are coloured by the four developmental time points Pink=NP, Dark Green=G, Light Green=L, Purple=PI.
\textbf{(c)} Dendrogram of clusters based on the mean expression values of the 15 clusters. The tree was computed based on Spearman's rank correlation with Ward linkage.
\textbf{(d)} Same as \textbf{(b)} but colouring cells by clusters.
\textbf{(e)} Heatmap highlighting some key marker genes to distinguish the main classes of clusters.
Colour scale represents log-transformed and normalized UMI counts scaled to a maximum of 1 per row.
The upper panel shows genes that were used to distinguish between luminal and basal cells.
Upper bars represent the cluster assignment and stages for the individual cells.
For visualization purposes only 100 randomly selected cells were shown for large clusters.
%-------------------------------------------------------------------------------------------------------------------


%---------------------------------------------Figure 2--------------------------------------------------------------
\textbf{Figure 2. Computational reconstruction of differentiation processes in the mammary gland}
\fakefigure{F2}
\textbf{(a)} Diffusion map of epithelial cells from the NP and G time points, showing the first three diffusion components.
\textbf{(b)} Differentiation trajectory of the luminal compartment based on the first two diffusion components.
\textbf{(c)} Same plot as in \textbf{(b)}, coloured by the normalized and scaled expression values of various genes.
%-------------------------------------------------------------------------------------------------------------------


%---------------------------------------------Figure 3--------------------------------------------------------------
\textbf{Figure 3. Pseudotime ordering identifies genes associated with luminal differentiation}
\fakefigure{F3}
\textbf{(a)} Definition of the hormone-sensing and secretory differentiation branch.
Cells are coloured by their progression through pseudotime, where low values represent undifferentiated cells.
\textbf{(b-d)} Examples of transcription factors with pseudotime-dependent expression with the same overall trend on both branches \textbf{(b)} or branch specific trends \textbf{(c,d)}.
\textbf{(e-f)} Heatmap of all genes with branch specific, pseudotime-dependent expression for the hormone-sensing lineage \textbf{(e)} or the secretory lineage \textbf{(f)}.
Pseudotime and the cluster assignment are annotated above the heatmap.
The values in the heatmap represent z-scaled, spline-smoothed expression values.
Genes in the heatmaps were clustered using hierarchical clustering with a dynamic tree cut.
%-------------------------------------------------------------------------------------------------------------------


%---------------------------------------------Figure 4--------------------------------------------------------------
\textbf{Figure 4. The effect of parity on the transcriptomic landscape of the luminal progenitor compartment}
\fakefigure{F4}
\textbf{(a)} Comparison of fold changes from Lp-NP versus luminal compartment and fold changes from Lp-PI versus the luminal cells.
The genes represent the top 500 differentially expressed genes between Lp-NP and luminal cells.
\textbf{(b)} Volcano plot illustrates differential expression between Lp-NP and Lp-PI, coloured dots represent significant genes with known function in lactation and immunity, dashed lines highlight the P value threshold of 0.01 and a log fold change of 1.
P values are adjusted for multiple testing using Benjamini-Hochberg.
\textbf{(c-d)} Visualization of expression difference for genes related to lactation \textbf{(c)} or immunity \textbf{(d)} for the six luminal clusters in NP and PI. Expression values correspond to normalized UMI counts.
\textbf{(e)} Cells from PI mapped onto the diffusion map from the NP and G time-point. Cells from NP and G are colored in grey.
%-------------------------------------------------------------------------------------------------------------------

\textbf{Table 1. Summary of mammary epithelial cell clusters}
\faketable{T1}
Overview of the different clusters including number of cells captured for each time-point and key genes that were used to infer their identities.
\textbf{Missing!}

\textbf{Supplementary Figure 1. Estrus cycle}
Vaginal smears of the mice from the NP and PI timepoints. The animals were classified as being in estrus (NP1, NP2, PI1) or diestrus (PI1). The scale bar represents 200$\mu m$.

\textbf{Supplementary Figure 2. Gating strategy to isolate MECs}
\textbf{(a)} Gating strategy used to select live, lineage\textsuperscript{negative} singlets.
\textbf{(b)} Representative plots for each of the four timepoints are shown. FSC-W: forward scatter width, FSC-A: forward scatter area, BSC-A: backward scatter area.

\textbf{Supplementary Figure 3. Quality control of sequencing data}
\textbf{(a)} Table summarizing quality control criteria per sample.
Number of unique molecules, genes detected and number of reads represent the median value for each sample.
\textbf{(b-c)} Histograms for the four conditions showing the distributions of the number of genes detected \textbf{(b)} or total number of molecules \textbf{(c)} metric.
\textbf{(d)} Scatterplot of number of genes detected versus percentage of mitochondrial RNA molecules.
For choice of thresholds see materials and methods.

\textbf{Supplementary Figure 4. Biological replicates account for little structure in the data.}
\textbf{(a)} t-SNE plot coloured by the eight different samples. All biological replicates are well mixed and none of the identified clusters is composed of cells from only one replicate.
\textbf{(b-c)} t-SNE plots highlighting the expression of luminal \textbf{(b)} and basal \textbf{(c)} genes in all 25,010 cells.

\textbf{Supplementary Figure 5. Clustering strategy}
\textbf{(a)} First round of clustering using an SNN-Graph and modularity maximization for cluster identification.
\textbf{(b)} Second round of clustering using hierarchical clustering based on Spearman's rank correlation with average linkage.
The gap statistic was used as optimization criterion to choose $k$.
\textbf{(c-e)} Marker genes that were used to remove immune \textbf{(c)} and endothelial \textbf{(d)} cells from the analysis. Cluster 10.2 (Prc) showed some degree of expression for some pericyte markers \textbf{(e)}.

\textbf{Supplementary Figure 6. Robustness of bifurcation event in diffusion map}
\textbf{(a)} The differentiation trajectory as determined by Monocle (see materials and methods) coloured by cluster assignment.
\textbf{(b)} The diffusion map is robust towards down-sampling of cells (100, 50 or 25\% of all cells were used in the left, middle or right panel, respectively) as well as the method of feature selection (all= all genes with mean expression level above 0.1, HVG= highly variable genes, PCA= first 50 components of PCA, selected= a manually selected choice of genes that are known to be involved in luminal cell differentiation).
The gene list included: \textit{Csn2, Gata3, Prlr, Elf5, Esr1, Pgr, Aldh1a3, Wap, Tspan8, Krt18, Krt8, Areg, Fgfr1, Fgfr2, Notch1, Notch3, Foxc1} and \textit{Zeb2}.

\textbf{Supplementary Figure 7. Gene set enrichment analysis of pseudo-time dependent genes}
\textbf{(a)} Significantly enriched GO-Terms for genes in the two clusters of the hormone-sensing branch.
\textbf{(b)} Significantly enriched GO-Terms for genes in the four clusters of the secretory branch.

\textbf{Supplementary Figure 8. Lp-PI is a post-parity luminal progenitor cluster}
\textbf{(a)} Number of differentially expressed genes between the NP and PI clusters of Hsd, Hsp and Lp.
\textbf{(b)} Principal component analysis (PCA) on all luminal cells from the PI time-point.
The PCA was computed on the top 500 differentially expressed genes between Lp and the rest of the luminal NP gland.
PC1 separates the progenitors and the differentiated cells, with the progenitors showing negative PC1 values.
\textbf{(c)} Genes that are higher expressed in Lp compared to the rest of the NP gland also have negative PC1 loadings.
\textbf{(d)} Top GO-terms (biological processes) that are significantly enriched in upregulated genes from \textbf{(a)}.
Dashed line indicates P value threshold at 0.001.

\textbf{Supplementary Figure 9. Removal of doublet clusters}
\textbf{(a)} Frequency of each cluster in all 8 samples, the threshold of 7\% is indicated by the red dashed line.
\textbf{(b)} Correlation of Bsl-G2 (on the y-axes) with Avd-G (left panels), Bsl-G (center panel) and the mean of Avd-G and Bsl-G (right panel) for both samples G1 and G2.
\textbf{(c-d)} Number of genes detected \textbf{(c)} and total number of molecules \textbf{(d)}. The values for each cell were normalized to the median value of the sample in which the cell was captured.

\textbf{Table S1.} Differentially expressed genes between clusters

\textbf{Table S2.} List of genes with pseudo-time dependent gene
expression with same overall trend in the two branches

\textbf{Table S3.} List of genes with pseudo-time dependent gene
expression with different trends in the two branches

\textbf{Table S4.} Genes differentially expressed between Lp-PI and Lp-NP

\section{Materials and methods}

\textbf{Animals}

All experimental animal work was performed in accordance to the Animals (Scientific Procedures) Act 1986, UK and approved by the Ethics Committee at the Sanger Institute.
C57BL/6N mice were housed in individually ventilated cages under a 12:12~h light-dark cycle, with water and food available \textit{ad libitum}.
The experiment was set up to allow for all of the developmental time points to be collected and tissues to be processed at the same time.
Mice were euthanized by terminal anaesthesia.
Females were mated with studs and allowed to litter.
Tissues were then harvested at gestation day 14.5 (G), lactation day 6 (L) and day 11 post natural weaning of the pups (PI).
Tissue from NP females was harvested at 8 weeks of age.
Two individual mice per developmental time point were used in the study.

\textbf{Mammary gland dissociation into single-cell suspension}

Lymph node divested mammary glands (excluding the cervical pair) were dissected from the mice and mechanically dissociated.
The finely minced tissue was transferred to a digestion mix consisting of DMEM/F12 (Gibco) + 10 mM HEPES (Gibco) + \SI{2}{\milli\gram\per\milli\litre} collagenase (Roche) + 200 U ml\textsuperscript{-1} hyaluronidase (Sigma) + \SI{50}{\micro\gram\per\milli\litre} gentamicin (Gibco) for 3 hours at \SI{37}{\celsius} and vortexed every 30 minutes.
After the lysis of red blood cells in NH\textsubscript{4}Cl, cells were briefly digested with warm 0.05\% Trypsin-EDTA (Gibco), 5 mg ml\textsuperscript{-1} dispase (Sigma) and 1 mg ml\textsuperscript{-1} DNase (Sigma), and filtered through a \SI{40}{\micro\meter} cell strainer (BD Biosciences).

\textbf{Cell labelling, flow cytometry and sorting}

Single cell suspensions were incubated in HF medium (Hank's balanced salt solution (Gibco) + 1\% foetal bovine serum, Gibco) + 10\% normal rat serum (Sigma) for 20 min on ice to pre-block before antibody staining.
All antibody incubations were performed for 10 min on ice in HF media.
Mammary cells were stained with the following primary antibodies: \SI{1}{\micro\gram\per\milli\litre} CD31-biotin (eBioscience, clone 390); \SI{1}{\micro\gram\per\milli\litre} CD45-biotin (eBioscience, clone 30F11); \SI{1}{\micro\gram\per\milli\litre} Ter119-biotin (eBioscience, clone Ter119) and \SI{0.5}{\micro\gram\per\milli\litre} EpCAM-PE (Biolegend, clone G8.8).
Cells were then stained with \SI{0.4}{\micro\gram\per\milli\litre} streptavidin-PE-CF594 (BD-Biosciences).
Propidium Iodide (PI, \SI{1}{\micro\gram\per\milli\litre}; Sigma) was used to detect dead cells.
Cells were filtered through a \SI{40}{\micro\meter} cell strainer (Partec) before sorting.
Sorting of cells was done using a SH800Z sorter (SONY).
Single-stained control cells were used to perform compensation manually and unstained cells were used to set gates.
Chip alignment and sorting calibration was performed with automatic setup beads (SONY) immediately prior to sorting.
Doublets, dead cells and contaminating haematopoietic, endothelial and stromal cells were gated out and EpCAM positive cells were sorted in LoBind\textsuperscript{\textregistered} microcentrifuges tubes (Eppendorf) with \SI{100}{\micro\litre} HF.
After sorting cells were spun down and resuspended in \SI{50}{\micro\litre} of HF. 
Samples were manually counted using an improved Neubauer chamber and the cell concentration was normalized by addition of HF.
Equal numbers of cells per sample were processed for scRNA library preparation.
Samples were processed for scRNA library preparation within 9 hours from tissue isolation.

\textbf{Library preparation and sequencing}

Library preparation was performed according to instruction in the 10X chromium single cell kit.

\textbf{RNA-seq data processing }

Read processing was performed as previously reported\autocite{Zheng2017}.
Briefly, the Cell Ranger Single-Cell Software Suite was used for demultiplexing, barcode assignment and UMI quantification (\href{http://software.10xgenomics.com/single-cell/overview/welcome}{\textit{http://software.10xgenomics.com/single-cell/overview/welcome}}).
The reads were aligned to the mm10 reference genome using a pre-built annotation package obtained from the 10X Genomics website.
All lanes per sample were processed using the ``cellranger count'' function.
The output from different lanes were then aggregated using ``cellranger aggr'' with --normalize set to ``none''.

\textbf{Quality control and preprocessing}

In total the Cell Ranger software identified 25,806 barcodes that contained enough unique molecules to be considered as cells (4,376 in NP, 6,021 in G, 9,603 in L and 5,806 in PI).
Libraries prepared from all time points showed high quality that was reproducible between the two biological replicates (Supplementary 2a).
We used the following metrics to flag poor quality cells: number of genes detected, total number of unique molecular identifiers (UMIs) and percentage of molecules mapped to mitochondrial genes.
Poor quality cells were then identified by setting a threshold on the number of genes and number of UMIs that was defined as three median absolute deviations (MAD) below the median for each sample, while requiring a minimum value of 1,000 total molecules and 500 genes detected.  
This resulted in the following thresholds for total number of genes detected: 1,042 for NP, 836 for G, 500 for L and 759 for PI; and the following for total number of molecules : 2,012 in NP, 1,479 in G, 1,000 in L and 1,379 in G.
In addition, all cells with 5\% or more of UMIs mapping to mitochondrial genes were defined as non-viable or apoptotic and removed from the analysis (Supplementary Figure 2d).
We finally also ensured that none of the reads in our dataset derived from index swapping\autocite{Sinha2017, Griffiths2017}.
For this we excluded cells with barcodes that appeared in more than one sample (non-unique barcodes).
The reasoning behind this being that any index swapped read between samples will also share the same cellular barcode.
Exclusion of these cells hence offers protection against index swapping.
This left us with a total number of 25,010 cells (4,223 in NP, 5,826 in G, 9,319 in L and 5,642 in P).
Gene expression values were then normalised by size factors, that were estimated as previously described\autocite{Lun2016}.
The log-transformed (log2(counts+1)) counts of highly variable genes (HVGs) were used as features for dimensionality reduction and clustering.
HVGs were identified by first fitting a mean-dependent trend to the gene-specific variances to all genes assuming that this trend is dominated by technical variance.
This trend was then used to estimate the technical component of the variance and all genes with a biological component (the residual variance) of at least 0.5 and a Benjamini-Hochberg adjusted P value smaller than or equal to 0.05 were defined as HVG.
The t-SNE embeddings were computed using the ``Rtsne'' package with default settings and perplexity set to 50 (\href{https://github.com/jkrijthe/Rtsne}{\textit{https://github.com/jkrijthe/Rtsne}}).

\textbf{Clustering}
Due to the large number of cells we used a two-step approach to identify clusters of cells.
In a first step we clustered cells using a shared-nearest-neighbor graph (SNN-graph) based approach.
This has the advantage of not requiring the computation of a distance matrix which is computationally prohibitive for large numbers of cells.
The SNN-graph was constructed using the \texttt{buildSNNGraph} function from scran with the number of nearest-neighbors set to 20.
Community-based clustering was then performed on the SNN-graph by multi-level modularity optimization using the \texttt{cluster\_louvain} function in igraph\autocite{Csardi2006}.
This identified a total of 13 clusters.
Despite this, the majority of clusters still contained substructure as evident from the t-SNE (Supplementary Figure 5a).
To resolve the cellular heterogeneity further we then applied agglomerative hierarchical clustering on each of the 13 clusters (Supplementary Figure 5b).
For this, we first computed the pairwise distances between cells based on Spearman's rank correlation of log-transformed HVG counts.
The dissimilarity matrix was then used to perform hierarchical clustering with the \texttt{hclust} function in R using average linkage.
Clusters were defined using the \texttt{cutree} function and the optimal $k$ was determined by maximizing the gap statistic using the \texttt{clusGap} function from the cluster package\autocite{Maechler2017}.
With this we identified a total number of 21 clusters.
As the last step we performed a \textit{post-hoc} test as previously reported\autocite{Macosko2015} by merging clusters with less than 10 differentially expressed genes (at a P value threshold of 0.01 and minimum log fold change of 1), resulting in one merged cluster (Myo). 
Next, we flagged clusters that expressed clear markers of non-epithelial cells as ``contaminating cells'' and removed them from the downstream analysis.
Clusters 6.1 and 6.2 were tagged as immune cells based on the expression of \textit{Cd74}, \textit{Cd72} and \textit{Cd54}\autocite{Scheele2017} and cluster 10.3 as endothelial cells based on the expression of \textit{Eng}, \textit{S1pr1} and \textit{Emcn} (Supplementary Figure 5c, Supplementary Figure 5d).
We retained cluster 10.2 as it expressed markers of a previously described celltype in the gland\autocite{Wang2015}, despite being positive for 2 out of 4 (\textit{Des} and \textit{Cspg4}) pericyte markers\autocite{Armulik2011} (Supplementary Figure 5e).
The high number of cells loaded on the 10x increased the chance of doublet formation in the droplet encapsulation\autocite{Zheng2017}, \fixme{which can result in artificial clusters}.
We thus identified clusters that appeared at low frequencies (less than the maximum doublet rate of 7\%, Supplementary Figure 9b) and that were highly similar to at least two other clusters present in the same sample ($\rho > 0.9$ of mean log2(counts+1)).
The only cluster that fulfilled this criterion was cluster Bsl-G2, whose gene expression was strongly correlated with Bsl-G and Avd-G (Supplementary Figure 9b)
In addition, the gene expression profile of this cluster is virtually identical to the average expression values of Bsl-G and Avd-G, supporting the hypothesis that it is indeed a mixture of the two clusters (Supplementary Figure 9b).
As Bsl-G and Avd-G are the most prevalent clusters in these samples (Table 1), doublets would be expected to be enriched for a combination of the two.
Finally, Bsl-G2 also belonged to the clusters with a high number of genes detected and a high number of unique molecules, which is also indicative for doublet clusters (Supplementary Figure 9c).
We hence excluded Bsl-G2 from any downstream analysis.


\textbf{Differential expression analysis }

Differential gene expression analysis was performed using ``edgeR''\autocite{Robinson2010}.
\fixme{Genes with a mean expression level below 0.1 counts were removed from the analysis.}
A negative binomial generalized log-linear model was fitted to the remaining genes with the cluster assignments as covariate(s).
The 'glmTreat' function was used to identify genes that have a significantly higher log fold change than 1 at an FDR of 0.01.

\textbf{Diffusion maps and pseudotime inference}

For inferring the differentiation trajectory we used diffusion maps.
First, we selected all cells from the NP and G time point (\autoref{F2}a) and detected the HVGs as described above.
The log-transformed (log2(count+1)) gene counts were then used to compute the first twenty diffusion components using the \texttt{DiffusionMap} function (default parameters as in 'destiny'\autocite{Angerer2016}).
After inspecting the fraction of explained variation for the first twenty components, we decided to retain the first 3.
In \autoref{F2}b we then focused on the luminal compartment and recomputed the diffusion map based only on the luminal cells, using the aforementioned procedure.
\fixme{Here we retained only the first two components (Supplementary Fig 3a).
Notably, the structure inferred by the diffusion map algorithm was robust to the choice of features and down-sampling of cells (Supplementary Fig 3d).
The structure of a common origin and the two branches could also be inferred using Monocle with standard settings\autocite{Trapnell2014} (Supplementary Fig 3c).
}
For inferring the branches and pseudotime ordering, we defined the following three tips, the cell with the smallest value for the second eigenvector (which was set as root) and the cells with the largest and smallest values for the first eigenvector (compare \autoref{F2}b).  

\textbf{Pseudotime-dependent expression}

To identify genes whose expression was significantly associated with the pseudotime we first fitted a natural cubic spline with three degrees of freedom to the log-transformed (log2(counts+1)) expression data in each branch.
A likelihood ratio test was then used to assess statistical significance of the fit compared to a null (pseudotime-independent) model.
Genes with a Benjamini-Hochberg corrected P value below 0.01 and a minimum change in log2(expression) along pseudotime of 0.5 were considered to be significantly pseudotime-dependent.
We then used a heuristic definition of branch-specific expression instead of modelling the branch assignment explicitly.
This was motivated as follows.
We were interested in general expression trends of genes, i.e. increase or decrease along the differentiation towards one of the two cell types, rather than comparing the exact timing of gene activation/inactivation between the branches.
Any approach trying to do the latter would have been complicated by the different cell densities along the branches, differences in branch length and by the difficulty of verifying any such hypotheses \textit{in vivo}.
Hence, we defined genes to be branch-specific when they were pseudotime-dependent in their expression in at least one of the two branches and when the gradient differed in signs between two branches.
The gradient was determined as the coefficient of a linear model fit to the spline-smoothed expression values, which was set to 0 if the coefficient was not significantly different from 0 at alpha=0.01.
Consequently, the gradient of a gene could either be -1 (decreasing), 0 (flat) or 1 (increasing).

\textbf{Gene set enrichment analysis}

A gene set enrichment analysis based on gene-ontology (GO) terms was conducted to characterize various genesets in the analysis.
The genes of interest were compared to all genes that were tested for differential expression using topGO with default settings\autocite{Alexa2016}.

\textbf{Code availability}

\fixme{All computational analyses were performed in R (Version 3.3.3) using standard functions unless otherwise indicated.
Code is available online at \href{https://github.com/MarioniLab/MammaryGland}{\textit{https://github.com/MarioniLab/MammaryGland}}.
Data can be accessed at this link (\url{https://karstenbach.shinyapps.io/webapp/}).
}

\textbf{Author contribution}

K.B. performed the experiments and all the computational analysis.
S.P. and K.B. setup and collected the mammary epithelial cells.
M.G. and J.H performed the 10X library production and sequencing.
J.C.M, D.A., K.B and W.T.K conceptualised the study and wrote the manuscript.

\textbf{Acknowledgements}

We would like to thank the staff at Sanger Institute, Research Service Facility (RSF) for their assistance.
We would like to thank Dr. Aaron T. Lun (CRUK CI) for helpful discussions and comments on the manuscript.
K.B. is funded by a Cambridge Cancer Centre studentship.
D.A. is funded by CRUK and Wellcome Trust.
S.P. is funded by CRUK.
J.C.M is funded by CRUK and EMBL.
W.T.K is funded by a CRUK career establishment award (C47525/A17348), University of Cambridge and Magdalene College, Cambridge.

\printbibliography
\end{document}
